% Opcje klasy 'iithesis' opisane sa w komentarzach w pliku klasy. Za ich pomoca
% ustawia sie przede wszystkim jezyk i rodzaj (lic/inz/mgr) pracy, oraz czy na
% drugiej stronie pracy ma byc skladany wzor oswiadczenia o autorskim wykonaniu.
\documentclass[inz, english, shortabstract]{iithesis}

\usepackage[utf8]{inputenc}

%%%%% DANE DO STRONY TYTUŁOWEJ
% Niezaleznie od jezyka pracy wybranego w opcjach klasy, tytul i streszczenie
% pracy nalezy podac zarowno w jezyku polskim, jak i angielskim.
% Pamietaj o madrym (zgodnym z logicznym rozbiorem zdania oraz estetyka) recznym
% zlamaniu wierszy w temacie pracy, zwlaszcza tego w jezyku pracy. Uzyj do tego
% polecenia \fmlinebreak.
\polishtitle    {Wymagający złamania wierszy\fmlinebreak tytuł pracy w~języku polskim}
\englishtitle   {Implementation of static and dynamic semantics for a calculus with algebraic effects and handlers using PLT Radex} 
\polishabstract {Abstrakt w języku polskim}
\englishabstract{English abstract}
% w pracach wielu autorow nazwiska mozna oddzielic poleceniem \and
\author         {Maciej Buszka}
% w przypadku kilku promotorow, lub koniecznosci podania ich afiliacji, linie
% w ponizszym poleceniu mozna zlamac poleceniem \fmlinebreak
\advisor        {dr hab. Dariusz Biernacki}
%\date          {}                     % Data zlozenia pracy
% Dane do oswiadczenia o autorskim wykonaniu
%\transcriptnum {}                     % Numer indeksu
%\advisorgen    {dr. Jana Kowalskiego} % Nazwisko promotora w dopelniaczu
%%%%%

%%%%% WLASNE DODATKOWE PAKIETY
%
%\usepackage{graphicx,listings,amsmath,amssymb,amsthm,amsfonts,tikz}
%
%%%%% WŁASNE DEFINICJE I POLECENIA
%
%\theoremstyle{definition} \newtheorem{definition}{Definition}[chapter]
%\theoremstyle{remark} \newtheorem{remark}[definition]{Observation}
%\theoremstyle{plain} \newtheorem{theorem}[definition]{Theorem}
%\theoremstyle{plain} \newtheorem{lemma}[definition]{Lemma}
%\renewcommand \qedsymbol {\ensuremath{\square}}
% ...
%%%%%

\begin{document}

%%%%% POCZĄTEK ZASADNICZEGO TEKSTU PRACY

\chapter{Introduction}
\begin{itemize}
  \item Algebraic effects and handlers
  \begin{itemize}
    \item What are algebraic effects
    \item Why do we need handlers
    \item 
  \end{itemize}
  
  \item Reduction semantics
  
  \item Type inference
  
  \item PLT Redex
  \begin{itemize}
    \item Defining languages, relations on terms, reduction semantics
    \item Automatic testing, counterexample search
  \end{itemize}
\end{itemize}

Algebraic effects (Plotkin ...) are an increasingly % TODO add ref
popular technique of structuring computational effects. They allow for seamless
composition of multiple effects, while retaining (unlike monads) applicative style
of programs. Coupled with handlers (Plotkin ...) which give programmers ability to
interpret effects, they provide a great tool for abstracting over a set of operations
which some program may perform and separating this interface from semantics of those
operations defined as effect handlers.

While designing a calculus representing a core of programming language, one must be very
careful deciding which features to add and keep in mind all the interactions between them. Following
most of the functional language designers I chose $\lambda$-calculus as a basis extended with operation
invocation, effect lifting and handling. The type system for this calculus is defined in Curry style,
which means that expressions do not contain types, which are assigned to them by the typing relation.
In more practical terms I had to implement type inference, instead of type checking. Also besides the
type system which specifies static semantics of calculus, one must specify the behavior of programs
at runtime. I chose the format of reduction semantics, as it is quite readable and is supported by
PLT Redex library. 

%%%%% BIBLIOGRAFIA

%\begin{thebibliography}{1}
%\bibitem{example} \ldots
%\end{thebibliography}

\end{document}
