\documentclass[inz, english, shortabstract]{iithesis}
%
% \usepackage[utf8]{inputenc} 
%
\polishtitle    {Implementacja statycznej i dynamicznej semantyki rachunku z efektami algebraicznymi i ich obsługą z pomocą biblioteki \Redex}
\englishtitle   {Implementation of static and~dynamic semantics for a calculus with algebraic effects and~handlers using \Redex} 
\polishabstract {Abstrakt w języku polskim}
\englishabstract{English abstract}
%
\author         {Maciej Buszka}
%
\advisor        {dr hab. Dariusz Biernacki}
%\date          {}                     % Data złożenia pracy
% Dane do oświadczenia o autorskim wykonaniu
%\transcriptnum {}                     % Numer indeksu
\advisorgen     {dr. hab. Dariusza Biernackiego}
\let\lll\undefined
\usepackage{csquotes, amsmath, amssymb, biblatex, float, ebproof}
\usepackage{minted}
%
\usepackage[xetex]{graphicx}
\addbibresource{mybib.bib}
%
\floatstyle{boxed} 
\restylefloat{figure} 
%
\newcommand{\Redex}{\texttt{PLT Redex}}
\newcommand{\Racket}{\texttt{Racket}}
\newcommand{\LC}{\(\lambda\)-calculus}
\mathchardef\mhyphen="2D % Define a "math hyphen"
\def\listingautorefname{Listing}
%
\begin{document}
%
\chapter{Introduction}
% dlaczego:
% experimentation with algebraic effects (calculus with various features)
% 
% lightweight prototyping
%
% goals:
% calculus
% implementation
% automatic testing (failed)
%
%
% What algebraic effects are, why are they interesting, why is it worth to experiment with different approaches/flavors
Algebraic effects \cite{Plotkin2003} are an increasingly popular technique of structuring computational effects.
They allow for seamless composition of multiple effects, while retaining (unlike monads) applicative style of programs.
Coupled with handlers \cite{Plotkin2013} which give programmers ability to interpret effects, they provide a disciplined and flexible tool for abstracting over a set of operations which a program may perform and for separating this interface from the semantics of those operations, defined as effect handlers.

% What are other approaches to algebraic effects
As composability and separation of concerns are often sought after, many calculi and languages have been developed in order to get algebraic effects just right; most notable of them being: \emph{Koka} \cite{Leijen2014} (featuring type inference, effect polymorphism with row-types and \emph{JavaScript}-like syntax), \emph{Links} \cite{Hillerstrom2016} (featuring \emph{ML}-like syntax, row-typed effect polymorphism and ad-hoc effects), \emph{Helium} \cite{Biernacki2019} (with abstract and local effects, \emph{ML}-style module system and principled approach to effect polymorphism), \emph{Eff} \cite{Bauer2012} (with implicit effect checking and recent work on direct compilation to \emph{OCaml} \cite{Kiselyov2018}) and \emph{Frank} \cite{Lindley2017} (with bidirectional type and effect system requiring minimal amount of effect variables, and shallow effect handlers). 

On a more theoretical side, various approaches to semantics of algebraic effects can be spotted in the literature, both with respect to type systems and run-time semantics.
Although most calculi use some form of row-types (with notable exception of \emph{Frank} \cite{Lindley2017}) to track effects, there are differences in permitted shapes (at most one effect \cite{Hillerstrom2016} of given type or many effects \cite{Biernacki2017,Leijen2014}, whether effects must be defined before use \cite{Biernacki2017,Lindley2017,Leijen2014,Bauer2012} or not \cite{Hillerstrom2016}) and how effects interact with polymorphism and abstraction.
At run-time handlers can wrap the captured continuation (giving so-called deep handlers \cite{Biernacki2017,Hillerstrom2016,Leijen2014,Bauer2012}) or not (shallow handlers \cite{Lindley2017}) and the very act of finding the right handler can be implemented in various ways, mainly depending on some constructs which skip handlers \cite{Biernacki2017}.

% What am I implementing, what is interesting in this thesis, how it differs from other systems/languages (The main goal of the thesis)
% Goals:
%  runnable calculus with algebraic effects
%  user friendly (usable, machine executable, algorithmic) type system
%  introspectable reductions
%  runnable abstract machine
%  language front-end facilitating ease of experimentation
%  some lightweight machine checking (automatic tests)
All this variety naturally invites us to experiment with different features and components of a calculus.
An executable model would be excellent for everyone interested in understanding inner workings of algebraic effects.
In particular, the ability to perform and visualize reductions would be very helpful for understanding of the dynamic semantics of operations and handlers.
In this thesis I will build such a calculus, describing my choices and discussing the trade-offs I faced.

The first goal of this thesis is the design of the calculus with effects and handlers, and its dynamic semantics.
Its implementation should be executable and allow for step-by-step reduction of calculus terms by a computer program.

The second goal is the design and implementation of a sound type system for the calculus, preferably with type inference.
It must enable easy use of effectful operations and allow for small yet complete examples.
The implementation should be runnable to facilitate explorability and learning by example.
It should also allow easy extension with additional language and type constructs.

The third goal is the design and implementation of abstract machine, which preserves the dynamic semantics of the calculus, yet is easier to translate into low-level virtual machine.
It could also be used as a basis for compilation to native code.

In order to rapidly iterate on the design and test the calculus, I decided to use the \Redex{} library which allows for building language model with executable type system judgments and reduction relation.
The calculus is designed to be small enough to be easily understood, yet have general language features to allow experimentation with reasonably complex programs.
The programmer can use $\lambda$-abstractions and recursive functions, numbers with addition, subtraction, multiplication and comparisons, booleans with conditional expressions and lists.

The algebraic effects are implemented with ad-hoc operations which take one parameter and return a value, handlers which can handle multiple (different) operations at once and lifts which allow operations to skip handlers.
The type system for calculus is presented in Curry style, with the implementation inferring simple types for unannotated terms using unification algorithm.
Although there is no way to create and bind polymorphic values, the system infers most general (simple) type for an expression which may contain unsolved variables.

The abstract machine is implemented with explicit stack of continuations and value environment.
It is given deterministic transition system using \Redex{}'s reduction relation and meta-function transforming a calculus expression into initial configuration.

Additionally to ease experimentation I implemented a language front-end which translates human-friendly programs to calculus terms, integrated with \Racket{} environment.

In brief summary, the development consists of:
\begin{itemize}
  \item Executable reduction semantics most similar to the system of \cite{Biernacki2017}.
  \item Curry style type system with ad-hoc effects in the style of \emph{Links}, effect rows based on \emph{Koka} and \emph{lift} construct of \cite{Biernacki2017}, implemented as a unification-based type inference algorithm.
  \item CEK style abstract machine with stack and \emph{meta}-stack of handlers, based on \cite{Hillerstrom2016}
\end{itemize}

% Outline of the thesis
The rest of this thesis is structured as follows: in the remainder of this chapter I introduce the main topics of this thesis, in \autoref{ch:calculus} I describe the calculus in greater detail, in \autoref{ch:implementation} I discuss technicalities of implementation and integration with the \Racket{} environment and \autoref{ch:manual} is user's manual.

\section{Algebraic effects and handlers}
Algebraic effects and handlers are a language level framework which allow for coherent presentation, abstraction, composition and reasoning about computational effects.
The key idea is to separate invocation of an effectful operation in an expression from the meaning of such an operation.
When one invokes an operation, current continuation (up to the nearest handler) is captured and passed along with the operation's argument to the nearest handler.
The handler in turn may execute arbitrary expression, using the continuation once, twice, returning a function which calls the continuation or simply ignoring it.
This way many control structures can be modeled and generalized by algebraic effects and appropriate handlers.
For example, the exceptions can be modeled using a single operation \texttt{Throw} and a handler which either returns the result when computation succeeded or returns a default value, ignoring the passed continuation.
\begin{verbatim}
handle e with
| Throw () r -> // return default value
| return x   -> x
end
\end{verbatim}
From the language design standpoint algebraic effects provide single implementation of various phenomena which may happen during execution of a program, for example mutable state, I/O, environment lookup, exceptions, etc., in a sense that every effect is treated the same, the typing rules are defined for invocation of any operation, and handling of any operation.
Similarly the operational semantics is also quite simple and succinct thanks to uniform treatment of various effects.
This framework is also extendable.
With small extension it can handle built-in effects in addition to user-defined ones.

From the language user perspective algebraic effects provide means of abstraction over effects used in a program.
Thanks to easy creation of new effects, one can define special purpose operations and their handlers to better represent domain specific problems while simultaneously using well known effects, defined in the standard library.
With effects being tracked by the type system, programmers can enforce purity or specific set of used effects at compile-time, or using effect polymorphism they can write reusable functions that abstract over effects which may happen.
The separation of definition and implementation of effects allows for various interpretations of operations, for example simulating a database connection or file-system during testing.
% TODO anonymous vs defined effects, abstract effects, effect polymorphism

\section{Types and type inference}
% The most common approach to giving an effectful computation a type uses a type-level data-structure known as a row.
% Initially developed in order to structurally type records, rows come in two flavors: Remy-style\cite{Remy1994} where they are treated as (finite) sets of label-type pairs, and Leijen-style\cite{Leijen2005} where rows are treated as (finite) lists of label-type pairs that are equivalent up to permutation of different labels.
% When polymorphism is present a row may have concrete prefix (possibly empty) and polymorphic tail denoted by a type variable.
% In effectful setting, the type system usually keeps a row of effects which an expression may perform, and suspended computations e.g. functions must have types decorated with a row of effects that may be performed when they are evaluated.
% In following listing, the function \texttt{sumPrintAndInc} would get a type $ Int \rightarrow (Get \, : \, Unit \Rightarrow Int , Set \, : \, Int \Rightarrow Unit , Print \, : \, Int \Rightarrow Unit) $ which notes all effects it may perform.
% The higher-order function \texttt{map} of type $ (a \rightarrow e \, b) \rightarrow List \, a $ would then 
% \begin{minted}{koka}
% let add = λ x Set +(Get (), x) in
% handle add 5 with
% | Set x r -> 
%   map sumPrintAndInc [1, 2, 3, 4]
% \end{minted}
Type inference is a technique of algorithmic reconstruction of types for various constructions used in a language.
It allows programmers to write programs with no type annotations, that often feel redundant and obfuscate the meaning of a program.
The most well known type system with inference is a system for \emph{ML} family of languages\cite{Pierce2002} -- \emph{Haskell}, \emph{OCaml}, \emph{SML} which infers the types with no annotations whatsoever.
A formal type system defines grammar of types consisting of base types (\texttt{int}, \texttt{bool} etc.), type constructors (arrows, algebraic data types) and type variables.
The typing rules require types which should be compatible (e.g. formal parameter and argument types) to unify.
The key feature of this system is the so-called let-polymorphism -- generalization of types of let-bound variables.
This way code reuse can be accomplished without complicating the type system and compromising type safety.
The basis of implementation of this system is first order unification algorithm\cite{Pierce2002}, which syntactically decomposes types and builds a substitution from type variables to types.



\section{Reduction semantics and abstract machines}
% TODO add reference to canonical definitions for good definition of reduction semantics and abstract machines, especially CEK
Reduction semantics\cite{Felleisen2009} is a format for specifying dynamic semantics of a calculus in an operational style.
The basic idea is to first define redexes -- expressions which can be reduced, and contexts in which the reduction can happen.
Taking \LC{} extended with numbers (\autoref{fig:lc-syntax}) and with call-by-value reduction order as an example, the only redex is application of a function to value $ (\lambda x . e) v $ as shown in \autoref{fig:lc-red}.
The possible contexts are: empty context $ \square $ or evaluation of operator part of application $ K e $ or evaluation of operand $ v K $ when left part has already been evaluated to a value.
\begin{figure}
  \includegraphics{lc-syntax.pdf}
  \caption{\LC{} abstract syntax}
  \label{fig:lc-syntax}
\end{figure}
With these possibilities in mind, we will define binary relation $ \longrightarrow $ which describes single step of reduction.
Such relation can be thought of as a transition system, rewriting terms into 'simpler' ones step by step.
There usually are two approaches to definition of such relation:
\begin{itemize}
  \item Definition of primitive reduction $ (\lambda x . e) v \longrightarrow_p e\{v/x\} $ which operates only on redexes and giving it a closure with the following inference rule:
  
  \begin{prooftree}
    \Hypo{e \longrightarrow_p e'}
    \Infer1{K[e] \longrightarrow K[e']}
  \end{prooftree}

  which says that if we can primitively reduce some expression, then we can do it in any context.
  \item Or definition of $ \longrightarrow $ directly, with decomposition of terms on both sides: $ K[(\lambda x . e) v] \longrightarrow K[e\{v/x\}] $
\end{itemize}
where the syntax $ e\{v/x\} $ means term $ e $ with value $ v $ substituted for variable $ x $, and $ K[e] $ means some context $ K $ with expression $ e $ inserted into the hole.
For both approaches it is important, that any term can be uniquely decomposed into redex and context, because when it is the case, then the relation is deterministic and gives good basis for formulation of abstract machines, interpreters or transformations to some other intermediate representations.
\begin{figure}
  \includegraphics{lc-red.pdf} 
  \caption{\LC{} reduction relation}
  \label{fig:lc-red}
\end{figure}
\begin{figure}
  \LC{} reduction example
  \caption{\LC{} example reduction sequence}
  \label{fig:lc-red-example}
\end{figure}

Abstract machine is a mathematical construction, usually defined as a set of configurations with deterministic transformations, which are computationally simple.
The goal for formulation of an abstract machine is to mechanize evaluation of terms while retaining semantics given in a more abstract format, e.g. reduction semantics, with the correspondence being provable\cite{Felleisen2009}.
As an example, \autoref{fig:lc-cek} shows a \emph{CEK}-machine for the \LC{} defined earlier.
The name \emph{CEK} comes from \emph{C}ommand, \emph{E}nvironment and \emph{K}ontinuation.
The machine configuration is a triple $ (e, \rho, \kappa) $ where $ e $ is an expression which is decomposed or reduced, $ \rho $ is an environment mapping variables to values, and the last component $ \kappa $ is a continuation stack, which determines what will happen with value, to which first component eventually reduces.
Thanks to the environment we no longer have to explicitly perform substitution, leading to more machine friendly and efficient implementation.
Given an initial state, the machine can then repeatedly apply transformation relation, either looping, arriving at a final value, or getting stuck. 
\begin{figure}
  \emph{CEK}-machine for \LC{}
  \caption{\LC{} abstract machine}
  \label{fig:lc-cek}
\end{figure}

\section{\Redex}
The \Redex{}\cite{Felleisen2009} library provides a comprehensive set of tools for the development of various calculi and language-like artifacts.
The work begins with the definition of a language using a familiar BNF-like syntax.
The library provides many options for defining patterns which describe the abstract syntax of the language, among them: meta-variables, symbols -- playing the role of markers, numbers, object language variables, repetitions of patterns using ellipsis and nonlinear patterns which can for example enforce that all variables in a binding are different.
Besides the syntax, the language definition allows for specifying the variable binding structure of the object language, which will be used by built-in meta-functions for substitution.
The following \autoref{lst:lc-typed-syntax-code} shows code extending \LC{} defined earlier in \autoref{fig:lc-syntax} with typed terms $ E $, along with types $ t $ and typing contexts $ \Gamma $.
\begin{listing}[H]
  \inputminted[firstline=26, lastline=32]{Racket}{../lc/lc.rkt}
  \caption{Typed \LC{} with numbers in \Redex{}}
  \label{lst:lc-typed-syntax-code}
\end{listing}
Second feature of the \Redex{} library are meta-functions which can pattern match on terms and return other terms.
They may use full power of complex and non-linear patterns which the library exposes, additional side conditions and even escape to \Racket{} -- host language in which the \Redex{} is defined.

\begin{listing}[H]
  \inputminted[firstline=34,lastline=51]{Racket}{../lc/lc.rkt}
  \caption{Type system for \LC{} in \Redex{}}
  \label{lst:lc-typed-type-code}
\end{listing}

The third aspect of this library are judgment forms which encode inductively defined judgments (e.g. type systems) with a syntax similar to pen-and-paper rules. \autoref{lst:lc-typed-type-code} shows example code, defining a simple type system for annotated \LC{} with numbers.
These rules can use patterns, meta-functions, other judgments and also escape to \Racket{}.
When defining a judgment, the programmer must specify which parameters are to be treated as inputs and which as outputs using \mintinline{Racket}{#:mode} keyword.
The library enforces an invariant that inputs \texttt{I} of a judgment form must be concrete terms and outputs \texttt{O} may be variables.
Under the hood the \Redex{} library will resolve the rules in depth-first order backtracking on failures and multiple pattern matches.

The last component of language modeled using \Redex{} are reduction relations, usually used for specifying semantics.
They are defined as a set of clauses, which should rewrite an input term into other term of same syntactic category.
In order to find redex, the programmer can define evaluation contexts, and then the library will decompose the terms using \mintinline{Racket}{(in-hole K e)} pattern, as in \autoref{lst:lc-red-code}.

\begin{listing}[H]
  \inputminted[firstline=19,lastline=23]{Racket}{../lc/lc.rkt}
  \caption{Reduction relation for \LC{} in \Redex{}}
  \label{lst:lc-red-code}
\end{listing}

Finally \Redex{} provides features for automatic testing via term generation facilities and has ability to typeset every component of calculus development.
Every figure in this thesis, which shows language grammar, reduction relation, meta-function or judgment has been generated using \Redex{}.

\chapter{The calculus}\label{ch:calculus}
The calculus implemented in this thesis is based on \LC{} with call-by-value semantics.
It's abstract syntax is presented in \autoref{fig:algeff-syntax}.
Meta-variable $ x $ ranges over variables used in value binders and their references, while $ op $ ranges over operation names, which are distinct from normal variables.
Meta-variable $ v $ ranges over values, which are one of: boolean $ b $, number $ m $, $ \lambda $-abstraction $ (\lambda \, x \, e) $ or recursive function $ (rec \, x \, x \, e) $.
Meta-variable $ e $ ranges over expressions, which include values $ v $, forms standard to \LC{} -- variables $ x $, function applications $ (e \, e) $, conditionals $ (if \, e \,e \, e) $ and primitive operations $ (prim \, e \ldots) $; expressions also include three constructs specific to effects -- operation invocations $ (op \, e) $, lifts $ (lift \, op \, e) $ and handlers $ (handle \, e \, hs \, ret) $ where $ ret $ is return expression $ (return \, x \, e) $ and $ hs $ is a list of handler clauses.

To achieve call-by-value, left-to-right reduction order I use evaluation contexts $ E $; this choice follows other calculi which allow for computational effects\cite{Biernacki2017, Leijen2014, Hillerstrom2016}.
One interesting aspect of these contexts is notion of \emph{free}ness\cite{Biernacki2017}, defined in \autoref{fig:algeff-free}.
The judgment $ free[ \, op , \, E , \, n] $ asserts that operation $ op $ is $ n $-free in evaluation context $ E $, meaning that it will be handled by $(n + 1)$st handler for $ op $ \emph{outside} the context $ E $.

The syntax of types, ranged over by meta-variable $ t $, comprises of base types ($Int$, $Bool$), arrow types $(t \rightarrow row \, t)$, operation types $(t \Rightarrow t)$, rows and type variables $ a $.
Rows are defined inductively as either empty row $\cdot$, variable $ a $ or extension $(op \,\, t \,\, row)$ of a row $row$ with type $t$ assigned to operation label $op$, and are ranged over by meta-variable $row$.
Finally, meta-variable $\Gamma$ ranges over typing contexts, $S$ over type substitutions and $SN$ denotes pair of substitution and name supply.

\begin{figure}
  \centering
  \includegraphics{algeff-syntax.pdf}
  \caption{Abstract syntax}
  \label{fig:algeff-syntax}
\end{figure}

\begin{figure}
  \centering
  \includegraphics{algeff-free.pdf}
  \caption{Context freeness}
  \label{fig:algeff-free}  
\end{figure}

\section{Dynamic semantics}
The dynamic semantics for a calculus with algebraic effects defines, besides the standard reductions known from \LC{}, the control structure of operations and handlers.
Intuitively, when an operation $op$ is invoked, it will be handled by dynamically closest handler, with a caveat that for each lift passed in search of handler, it must skip one handler.
Formally, when an operation $ op $ is invoked, it will be handled by lexically enclosing handler $ (handle \, E[op \, e] \, hs \, ret) $ if and only if the intermediate context $ E $ is $0$-free\cite{Biernacki2017}.

The dynamic semantics is defined in the format of contextual reduction semantics in \autoref{fig:algeff-red}.
All rules perform reduction in a context $ E $ leaving it unchanged.
First rule describes standard $ \beta $-reduction via substitution of argument value for variable in function body, while the second is the $ \beta $-reduction of recursive function, where first we substitute the function for function variable and then substitute the argument.
Next rule deals with built-in primitive operations, delegating to auxiliary function \textit{prim-apply} which pattern matches on $ prim $ and performs appropriate operation.
Next two rules perform choice of correct branch in conditional expression depending on value of condition.
The rule \textit{lift-compat} returns value from lift expression, leaving it unchanged.
The rule \textit{handle-return} handles the case when inner expression of handle expression evaluates to a value, which means we have to evaluate return clause by substituting the result value for $ x $, and plugging this expression into evaluation context $ E $.
The last rule \textit{handle-op} describes the behavior when an expression calls some operation.
To handle an operation we must find $0$-free inner context $ E_2 $ which is directly surrounded by handle expression which has a case for $ op $.
Then we substitute value $ v $ for first variable of operation handler and the inner context $ E_2 $ surrounded with the very same handler (the continuation delimited by the handler) closed in a lambda for the second argument.
This way the operation handler can resume the evaluation of expression which invoked the handled operation.

\begin{figure}
  \centering
  \includegraphics{algeff-red.pdf}
  \caption{Reduction relation} 
  \label{fig:algeff-red}
\end{figure} 

\section{Static semantics}
The type system is based on \emph{Koka}\cite{Leijen2014} (Leijen's style of row types\cite{Leijen2005}), \emph{Links}\cite{Hillerstrom2016} (ad-hoc operations) and Biernacki et al. \cite{Biernacki2017} (lift construct) systems.
Initially I implemented a variant of System-F extended with row-types but it proved to be a bit of a mouthful to write even simplest programs.
Additionally the \texttt{Redex}'s facilities for automatic testing were not able to generate sufficiently many well typed terms, so some type inference was inevitable.
To limit the amount of work I decided to present the calculus in Curry style, with typing relation inferring the type for unannotated terms.

Building on well known foundations\cite{Pierce2002}, types are inferred via first order unification.
While exact algorithm is presented in \autoref{sec:unification}, the notion of unification is used extensively in the remainder of this chapter and as such I will present intuitive definition here.
Two types $t_1$ and $t_2$ \emph{unify} (written $t_1 \sim t_2$) if they are structurally the same, where variables can be substituted with any type.
Two rows unify, if they are the same list of operation-type pairs, up to permutation of different operations.

The system does not feature polymorphism in first class fashion, as there is no rule where types are generalized, but I believe it to be a straightforward addition, following the \emph{Koka}\cite{Leijen2014} calculus.
Still, after inferring type of an expression, we can see which unification variables are left abstract and could be generalized.
There are two main features differentiating this system from \emph{Koka}'s; firstly effects need not be defined before use, their signature is inferred the same way as any other construction; secondly the system is algorithmic, with rules explicitly encoding a recursive function which can infer the type of an expression.

\subsection{Type inference}
The judgment $ \Gamma \mid [S_1 \, N_1] \vdash e \, : \, t \, ! \, row \mid [S_2 \, N_2] $ asserts that in typing context $ \Gamma $ under type substitution $ S_1 $, with name supply state $ N_1 $ expression $ e $ has type $ t $ with effects $ row $ under type substitution $ S_2 $ and with name supply state $ N_2 $.
Algorithmically this judgment infers a type and an effect row, and calculates new substitution, given typing environment, current substitution and an expression.
As in \emph{ML} languages only simple types can be inferred, along with effect rows.

\begin{figure}
  \centering
  \includegraphics{algeff-infer.pdf}
  \caption{Type system} 
  \label{fig:algeff-infer} 
\end{figure}

Base rules for constants and variable lookup are straightforward, each introducing fresh effect row variable.
To check $ \lambda $ expression, we first introduce fresh type variable, and then check the body in extended environment.
The arrow gets annotated with effects which may occur during evaluation of the body and the $ \lambda $ abstraction itself is returned with fresh effect row.
The recursive function checking is similar to normal functions.
First variable denotes function itself, while second its argument.
Accordingly, the environment gets extended with functional type $ t_1 \rightarrow row_1 t_2 $ and argument type $ t_1 $, to check the body of the function, and afterwards the result type of body $ t $ gets unified with the result type of function $ t_2 $, same with effect row.
Whole function, as it is a value, is returned with fresh effect row.
The application requires expression at function position to be of functional type and parameter type to unify with argument type.
All effect rows (from evaluation of function value, argument value and function body) must unify as well.
Inference for primitive operation call is deferred to auxiliary judgment, which checks arity and argument types, returning result type and usually fresh effect row.
Conditional expression requires the condition to be of type $ Bool $ and types of two branches to unify.
As usual all effect rows must also unify.
Operation invocation requires the effect row to contain operation $ op $ with type $ (t_1 \Rightarrow t_2) $ where input type $ t_1 $ is the inferred type for $ e $ and output type $ t_2 $ is fresh.
Operation lifting prepends fresh $ op $ to the effect row of $ e $.
Finally, to check handle expression we first infer the type of enclosed expression $ e $, then in environment extended with $e$'s type $ t_1 $ we infer return expression's type $ t_{ret} $.
Helper judgment \textit{infer-handlers} returns the result effect row of handlers $ row_{out} $ and row marking handled effects $ row_{handled} $ whose tail is the same as result's.
By unifying result row with return row and handled row with $ row_1 $ we ensure that effects which may occur during handling of operations, evaluation of return clause and leftovers from the inner expression are all accounted for.

\subsection{Inference for effect handlers}

\begin{figure}
  \centering
  \includegraphics{algeff-infer-handlers.pdf}
  \caption{Handlers type inference} 
  \label{fig:algeff-infer-handlers} 
\end{figure} 

List of effect handlers $ hs $ is processed right-to-left by judgment 
$$ infer\mhyphen handlers [\Gamma , \, SN_{in} , \, t_{ret} , \, hs , \, row_{out} , \, row_{handled} , \, SN_{out}] $$
The $ t_{ret} $ is the type of return clause, $ row_{out} $ is the combined row of effects which may occur in any handler and $ row_{handled} $ is the row of handled operations, with appropriate types.
The base case of empty list initializes both rows with the same type variable, this way $ row_{handled} $ returned by $infer\mhyphen handlers$ judgment will consist of all handled operations and its tail will be $ row_{out} $.
The inductive case first calculates $ row_{out} $ and $ row_{handled} $ for the tail of the list.
Then, two new fresh variables are created: $ t_v $ for the type of value passed to handler and $ t_r $ for type of resumption parameter.
Resumption's result type is the result type of return clause as it should eventually evaluate to a value, which will be transformed by that clause.
It's effect row is the same as the result row of whole handle expression, which means that any subsequent uses of operations will be handled by this handler.
With environment extended with $ t_v $ for first parameter -- an argument to handler expression and $ t_r \rightarrow row_{out} \, t_{ret} $ for second parameter -- the resumption, we check the handler expression $ e $.
We then unify the effects of evaluating $ e $ with all effects of handlers, and result type $ t_{h} $ with return type $ t_{ret} $.
Finally we extend handled row with current operation $ (op (t_v \Rightarrow t_r)) $

\section{Abstract machine}
Abstract machine is based on \emph{CEK}-machine of Hillerström and Lindley\cite{Hillerstrom2016}, with the difference that during the search for operation handler, machine must count handlers and lifts it passes.

\begin{figure}
  \centering
  \includegraphics{algeff-am-syntax.pdf}
  \caption{Abstract machine configurations}
  \label{fig:algeff-am-syntax}
\end{figure}

Possible configurations are given in \autoref{fig:algeff-am-syntax}.
Meta-variable $ C $ ranges over shapes of machine configurations, meta-variable $ V $ ranges over machine values, which are distinct from calculus value, e.g. function values must now keep their environment $ \rho $ which maps variables to machine values.
Additionally one of possible values is a \emph{meta-stack} which one may think of as a continuation captured during operation invocation.
Meta-variable $ \sigma $ defines normal (or pure) continuation frames and $ \Sigma $ denotes pure continuation \emph{stack}, while $ \phi $ ranges over effect frames -- \emph{handle}, \emph{lift} and \emph{done} token which marks final continuation.
Meta-variable $ \kappa $ denotes meta-frame which consists of pure stack and one effect frame.
Finally $ K $ ranges over stacks of meta-frames, which I will call \emph{meta-stack}s.
Meta-function \emph{initial-conf} in \autoref{fig:algeff-am-initial-conf} transforms an expression into initial configuration -- initializing machine with empty stack, '\emph{done}' effect frame and empty meta-stack.

\begin{figure}
  \centering
  \includegraphics{algeff-am-initial-conf.pdf}
  \caption{Abstract machine -- initial configuration}
  \label{fig:algeff-am-initial-conf}
\end{figure}

\begin{figure}[h]
  \centering
  \includegraphics{algeff-am-a.pdf} 
  \caption{Abstract machine -- administrative transitions}
  \label{fig:algeff-am-a}
\end{figure}

The first group of transitions depicted in \autoref{fig:algeff-am-a} performs mostly administrative functions -- capturing environment for functional values, transforming from value terms to machine values, sequencing binary operations, switching to special configuration for search of handler and a transition for returning final value.

\begin{figure}[h]
  \centering
  \includegraphics{algeff-am-b.pdf} 
  \caption{Abstract machine -- continuation building}
  \label{fig:algeff-am-b}
\end{figure}

Second group of transitions (\autoref{fig:algeff-am-b}) decomposes current expression and builds continuation by growing either stack or meta-stack.
First four rules -- for function application, operation invocation, primitive operation call and conditional expression -- all create a new pure frame and push it onto stack.
Last two transitions deal with effectful operations -- handle and lift; they build meta-stack by bundling previous effect frame with current stack into meta-frame, pushing it onto meta-stack and then installing fresh stack and new effect frame into machine configuration.

\begin{figure}[h]
  \centering 
  \includegraphics{algeff-am-c.pdf}
  \caption{Abstract machine -- contractions}
  \label{fig:algeff-am-c}
\end{figure}

Third group of transitions (\autoref{fig:algeff-am-c}) perform various reductions.
First rule looks up value of a variable in current environment, second rule performs contraction of normal function, by extending the environment with mapping from function's formal parameter $ x $ to calculated value $ V $.
Third rule reduces recursive function, extending environment with argument value $ V $ and function value, allowing for recursive calls.
Last rule handling application deals with continuation resumption, by pushing current stack and effect frame ($ [(\sigma \ldots) \, \phi_1] $) onto meta-stack, installing the top ($ [\Sigma \, \phi_2] $) of captured meta-stack and prepending its tail ($ (\kappa_1 \ldots) $) to meta-stack.

\begin{figure}[h]
  \centering
  \includegraphics{algeff-am-e.pdf}
  \caption{Abstract machine -- effect handling}
  \label{fig:algeff-am-e}
\end{figure}

Transitions in last group (\autoref{fig:algeff-am-e}) form essence of this machine, performing all tasks related to effect handling.
First four rules search for appropriate handler for $ op $ by maintaining a counter $ n $ which is incremented by every lift for $ op $ and decremented by every handler for $ op $.
Fifth rule matches when the counter is $ 0 $ and handler has a case for operation which was invoked. 
In this situation, the machine must begin evaluation of handling expression, first extending environment with passed argument and captured continuation with current meta-frame appended.
Two last rules deal with returning values -- in case of handler the return clause is installed, and in case of lift its frame is simply discarded.

\chapter{Implementation}\label{ch:implementation}
\section{Unification}\label{sec:unification}
The unification algorithm is loosely based on \cite{Pierce2002}, but instead of keeping a set of constraints to be solved, it solves sub-problems recursively.
It is implemented as a \Redex{} judgment, which takes two types, the initial substitution and returns the substitution extended with their unifier.
This judgment must also pass around the name supply token -- a natural number which is incremented every time a new type variable is created.
Variables in types are substituted lazily -- whenever algorithm encounters variable which is in domain of the substitution, it looks it up and continues unification.
The \autoref{fig:algeff-unify} shows the unification algorithm.

\begin{figure}[h]
  \centering
  \includegraphics{algeff-unify.pdf}
  \caption{Unification algorithm}
  \label{fig:algeff-unify}
\end{figure}

Terminal rules \emph{refl-var} and \emph{refl-const} handle unification of equal variables and constants respectively.
In a special case \emph{refl-var-lookup}, when both types are variables, the left variable $a_1$ is not in the domain of substitution and the right variable $a_2$ is mapped to $a_3$ which is equal to $a_1$, then they are in fact the same variable and should unify.
This corner-case arises due to lazy application of substitution to types.

The rule \emph{ext} extends substitution, when the left-hand-side type is a variable which is not in the domain of substitution.
It is important to note, that the type which is inserted into substitution is fully substituted.
This way we can maintain an invariant, that every type in the substitution has only type variables which are not in the domain of the substitution.
The last side condition, that $a$ is not a free type variable in $t$ is the occurs-check, which ensures that algorithm doesn't try to unify cyclic types, which would loop.

When the left type is a variable, in the domain of substitution, rule \emph{lookup} looks it up, and continues unification.
When the left type is not a variable, and right type is, rule \emph{flip} flips them around and continues unification.
This rule is needed because the judgment performs extension and lookup only on left variable.
The next three rules \emph{->}, \emph{=>} and \emph{List} decompose the type constructor and unify all respective sub-types.

The last rule \emph{row} unifies two rows.
It requires the left row to be non empty, and the right row not to be a variable (to ensure determinism with respect to the rule \emph{flip}).
Then it uses helper judgment \emph{unify-row} which rewrites $row_2$ such that its head is $op \, \, t_2$ and rest is bound $row_r$.
The side condition, requiring the variable at the end of (substituted) $row_1$ not to be in domain of substitution which was build by the rewriting ensures that algorithm does not try to unify rows with different labels, but same type variable in the tail (similar to occurs-check in rule \emph{ext}).

The helper judgment \emph{unify-row}, shown in \autoref{fig:algeff-unify-row} rewrites the row such that it begins with desired label $op$.
It is based on similar judgment in \cite{Leijen2005}.
There are two base cases: either the row already begins with $op$ (rule \emph{row-head}) and is decomposed and returned or the row is already a variable which is not bound by substitution (rule \emph{row-var}).
In this case the substitution is extended with a row $(op \, t \, row)$ where both $t$ and $row$ are fresh variables.

The third rule \emph{row-lookup} handles variables which are bound by the substitution, looking up the appropriate row and continuing row rewriting.
The last rule \emph{row-swap} matches rows that have a label $o1$ different to desired label $o$.
In this case, the rest of the row must rewritten recursively yielding type $t_2$ for $o$ and new tail.
The type is returned and the tail is extended with original head.

\begin{figure}
  \centering
  \includegraphics{algeff-unify-row.pdf}
  \caption{Row unification algorithm}
  \label{fig:algeff-unify-row}
\end{figure}

\section{Automatic testing}

\section{The \Racket{} environment}

% TODO implementation architecture
\chapter{User's manual}\label{ch:manual}

\printbibliography

%\begin{thebibliography}{1}
%\bibitem{example} \ldots
%\end{thebibliography}

\end{document}
