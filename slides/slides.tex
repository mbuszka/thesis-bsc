\documentclass{beamer}

\mode<presentation>
{
  \usetheme{Warsaw}      % or try Darmstadt, Madrid, Warsaw, ...
  \usecolortheme{default} % or try albatross, beaver, crane, ...
  \usefonttheme{default}  % or try serif, structurebold, ...
  \setbeamertemplate{navigation symbols}{}
  \setbeamertemplate{caption}[numbered]
} 
\usepackage[polish]{babel}
\usepackage{amsmath}
\usepackage{amssymb}
\usepackage{fancyvrb}
\usepackage{fontspec}

\setmonofont{CMU Typewriter Text}

\newcommand{\Redex}{\texttt{PLT Redex}}
\newcommand{\Racket}{\texttt{Racket}}
\newcommand{\LC}{\(\lambda\)-calculus}

\title[A calculus with algebraic effects and~handlers]{Implementation of static and~dynamic semantics for a calculus with algebraic effects and~handlers using \Redex}
\author{Maciej Buszka}
\institute{Instytut Informatyki UWr}
\date{15.02.2019}

\begin{document}

\begin{frame}
  \titlepage
\end{frame}

% Uncomment these lines for an automatically generated outline.
% \begin{frame}{Outline}
% 	\tableofcontents
% \end{frame}

\begin{frame}{Plan prezentacji}
  \tableofcontents
\end{frame}

\section{Wstęp}
\subsection{Cel}
\begin{frame}{Motywacja i cel pracy}
  \begin{itemize}
    \item Nowe, aktywnie rozwijane zagadnienie
    \pause
    \item Nietrywialna semantyka i implementacja
    \pause
    \item Brak interaktywnego modelu rachunku
    \pause
    \item Cel -- zaprojektowanie i implementacja:
    \begin{itemize}
      \pause
      \item Rachunku oraz jego semantyki dynamicznej
      \pause
      \item Semantyki statycznej
      \pause
      \item Maszyny abstrakcyjnej
    \end{itemize}
  \end{itemize}
\end{frame}

\subsection{Efekty algebraiczne}
\begin{frame}{Efekty algebraiczne}
  \begin{itemize}
    \item Kompozycja wielu różnych efektów obliczeniowych
    \pause
    \item Definicja własnych efektów
    \pause
    \item Rozdzielenie interfejsu i implementacji
    \pause
    \item Program może używać abstrakcyjnych operacji -- interfejs
    \pause
    \item Wyrażenie obsługujące -- implementacja
  \end{itemize}
\end{frame}

% \begin{frame}{Obsługa operacji}
%   \begin{itemize}
%     \item Normalny tok obliczeń jest przerwany
%     \pause
%     \item Odnalezione zostaje wyrażenie obsługujące (\emph{handler})
%     \begin{itemize}
%       \pause
%       \item Wartość przekazana przy wywołaniu operacji
%       \pause
%       \item Funkcja reprezentująca resztę obliczenia które zostało przerwane
%     \end{itemize}
%     \pause
%     \item Wznowienia może być użyte dowolnie wiele razy
%   \end{itemize}
% \end{frame}

% \subsection{Przykład 1}
% \begin{frame}[fragile]{Przykład 1}
%   \pause
% \begin{Verbatim}[commandchars=\\\{\}]
% (handle
%   (+ (op:Magic (- 2 1)) 1)
%   ((op:Magic (v:x v:r (app v:r 42))))
%   (return v:x -(v:x, 1))
% \end{Verbatim}
% \end{frame}

% \begin{frame}[fragile]{Przykład 1}
% \begin{Verbatim}[commandchars=\\\{\}]
% (handle
%   (+ (op:Magic \textcolor{orange}{(- 2 1)}) 1)
%   ((op:Magic (v:x v:r (app v:r 42))))
%   (return v:x -(v:x, 1))
% \end{Verbatim}
% \end{frame}

% \begin{frame}[fragile]{Przykład 1}
% \begin{Verbatim}[commandchars=\\\{\}]
% (handle
%   (+ (op:Magic \textcolor{blue}{1}) 1)
%   ((op:Magic (v:x v:r (app v:r 42))))
%   (return v:x -(v:x, 1))
% \end{Verbatim}
% \end{frame}

% \begin{frame}[fragile]{Przykład 1}
% \begin{Verbatim}[commandchars=\\\{\}]
% (handle
%   \textcolor{red}{(+} \textcolor{orange}{(op:Magic 1)} \textcolor{red}{1)}
%   ((op:Magic \textcolor{orange}{(v:x v:r (app v:r 42))}))
%   (return v:x -(v:x, 1))
% \end{Verbatim}
% \end{frame}

% \begin{frame}[fragile]{Przykład 1}
% \begin{Verbatim}[commandchars=\\\{\}]
% \textcolor{blue}{(app}
%   \textcolor{blue}{(λ v:z}
%     \textcolor{blue}{(handle}
%       \textcolor{red}{(+ }\textcolor{blue}{v:z} \textcolor{red}{1)}
%       \textcolor{blue}{(...)}
%       \textcolor{blue}{(return v:x -(v:x, 1))))}
%   \textcolor{blue}{42)}
% \end{Verbatim}
% \end{frame}

% \begin{frame}[fragile]{Przykład 1}
% \begin{Verbatim}[commandchars=\\\{\}]
% \textcolor{orange}{(app}
%   \textcolor{orange}{(λ v:z}
%     \textcolor{orange}{(handle}
%       \textcolor{orange}{(+ v:z 1)}
%       \textcolor{orange}{(...)}
%       \textcolor{orange}{(return v:x -(v:x, 1))))}
%   \textcolor{orange}{42)}
% \end{Verbatim}
% \end{frame}

% \begin{frame}[fragile]{Przykład 1}
% \begin{Verbatim}[commandchars=\\\{\}]
% \textcolor{blue}{(handle}
%   \textcolor{blue}{(+ 42 1)}
%   \textcolor{blue}{( ... )}
%   \textcolor{blue}{(return v:x«38» -(v:x«38», 1)))}
% \end{Verbatim}
% \end{frame}

% \begin{frame}[fragile]{Przykład 1}
% \begin{Verbatim}[commandchars=\\\{\}]
% (handle
%   \textcolor{orange}{(+ 42 1)}
%   ( ... )
%   (return v:x«38» -(v:x«38», 1)))
% \end{Verbatim}
% \end{frame}

% \begin{frame}[fragile]{Przykład 1}
% \begin{Verbatim}[commandchars=\\\{\}]
% (handle
%   \textcolor{blue}{43}
%   ( ... )
%   (return v:x«38» -(v:x«38», 1)))
% \end{Verbatim}
% \end{frame}

% \begin{frame}[fragile]{Przykład 1}
% \begin{Verbatim}[commandchars=\\\{\}]
% (handle
%   \textcolor{orange}{43}
%   ( ... )
%   \textcolor{orange}{(return v:x«38» -(v:x«38», 1))})
% \end{Verbatim}
% \end{frame}

% \begin{frame}[fragile]{Przykład 1}
% \begin{Verbatim}[commandchars=\\\{\}]
% \textcolor{blue}{-(43, 1)}
% \end{Verbatim}
% \end{frame}

% \begin{frame}[fragile]{Przykład 1}
% \begin{Verbatim}[commandchars=\\\{\}]
% \textcolor{orange}{-(43, 1)}
% \end{Verbatim}
% \end{frame}

% \begin{frame}[fragile]{Przykład 1}
% \begin{Verbatim}[commandchars=\\\{\}]
% \textcolor{blue}{42}
% \end{Verbatim}
% \end{frame}

\section{Rachunek oraz mikrojęzyk algeff}
\subsection{Rachunek}
% \begin{frame}{Rachunek}
%   \pause
%   Rachunek który zaimplementowałem udostępnia:
%   \begin{itemize}
%     \pause
%     \item Wyrażenia liczbowe z podstawowymi operacjami
%     \pause
%     \item Wyrażenia logiczne i warunkowe
%     \pause
%     \item Homogeniczne listy (dla dowolnego typu)
%     \pause
%     \item Funkcje anonimowe oraz rekurencyjne
%   \end{itemize}
% \end{frame}

\begin{frame}{Rachunek}
  \begin{itemize}
    \item Abstrakcyjne operacje
    \begin{itemize}
      \pause
      \item Nie muszą być zdefiniowane a priori
      \pause
      \item Przyjmują jeden argument, zwracają jedną wartość
    \end{itemize}
    \pause
    \item Wyrażenia obsługujące
    \begin{itemize}
      \pause
      \item Dostęp do wznowienia
      \pause
      \item Semantyka tzw. głębokiej obsługi
      \pause
      \item Obsługa wielu operacji naraz
      \pause
      \item Klauzula \texttt{return}
    \end{itemize}
    \pause
    \item Wyrażenia podnoszące (\emph{lift})
    \begin{itemize}
      \pause
      \item 'przeskoczenie' najbliższego wyrażenia obsługującego
    \end{itemize}
    \pause
    \item Wartości bazowe oraz operacje na nich
  \end{itemize}
\end{frame}

\subsection{Model}
\begin{frame}{Model}
  \begin{itemize}
    \item Składnia abstrakcyjna i semantyka dynamiczna
    \pause
    \item System typów
    \begin{itemize}
      \item Odtwarza typ i efekt wyrażenia
      \pause
      \item Inferowane są tylko typy proste
      \pause
      \item Polimorfizm nie jest wspierany
      \pause
      \item Najogólniejszy typ wyrażenia
    \end{itemize}
    \pause
    \item Maszyna abstrakcyjna
    \begin{itemize}
      \pause
      \item Wiele konfiguracji
      \pause
      \item Jawne środowisko
      \pause
      \item Stos i meta-stos
    \end{itemize}
  \end{itemize}
\end{frame}

\subsection{Przykłady}
\begin{frame}[fragile]{Przykład}
  Program:
  
  \VerbatimInput[numbers=left]{02.rkt}
  
  \pause
  Oczekiwany wynik:

  \texttt{42}
\end{frame}

\begin{frame}[fragile]{Przykład}
\begin{Verbatim}[numbers=left]
(app
  (handle
    (+ (op:Get 0) (op:Get (op:Set 29)))
    ((op:Get (v:i v:r (λ v:s (app (app v:r v:s) v:s))))
     (op:Set (v:s v:r (λ v:i (app (app v:r 0) v:s)))))
    (return v:x (λ v:s v:x)))
  13)
\end{Verbatim}
\begin{itemize}
  \item Składnia abstrakcyjna:
  \pause
  \begin{itemize}
    \item linia 1. app -- aplikacja
    \pause
    \item linia 3. wyrażenie zagnieżdżone
    \pause
    \item linie 4. i 5. klauzule obsługujące
  \end{itemize}
\end{itemize}
\end{frame}

\begin{frame}[fragile]{Przykład}
Wywołanie operacji \texttt{Get}
\begin{Verbatim}[commandchars=\\\{\},numbers=left]
(app
  (handle
    \textcolor{red}{(+} \textcolor{orange}{(op:Get 0)} \textcolor{red}{(op:Get (op:Set 29)))}
    ((op:Get \textcolor{orange}{(v:i v:r (λ v:s (app (app v:r v:s) v:s)))})
     (op:Set (v:s v:r (λ v:i (app (app v:r 0) v:s)))))
    (return v:x (λ v:s v:x)))
  13)
\end{Verbatim}
\begin{itemize}
  \pause
  \item kolor \textcolor{orange}{pomarańczowy} -- operacja i jej handler
  \pause
  \item kolor \textcolor{red}{czerwony} -- przechwycony kontekst
\end{itemize}
\end{frame}

\begin{frame}[fragile]{Przykład}
\begin{Verbatim}[commandchars=\\\{\},numbers=left]
(app
  \textcolor{blue}{(λ v:s«312»}
    \textcolor{blue}{(app} 
      \textcolor{blue}{(app}
        \textcolor{blue}{(λ v:z}
          (handle
            \textcolor{red}{(+} \textcolor{blue}{v:z} \textcolor{red}{(op:Get (op:Set 29)))}
            ( ... )
            (return v:x (λ v:s v:x)))\textcolor{blue}{)}
        \textcolor{blue}{v:s«312»)}
      \textcolor{blue}{v:s«312»))}
  13)
\end{Verbatim}
\begin{itemize}
  \item kolor \textcolor{red}{czerwony} -- przechwycony kontekst
  \item kolor \textcolor{blue}{niebieski} -- wyrażenie stworzone przez redukcję
\end{itemize}
\end{frame}

\begin{frame}[fragile]{Przykład}
$\beta$-redukcja
\begin{Verbatim}[commandchars=\\\{\},numbers=left]
\textcolor{orange}{(app}
  \textcolor{orange}{(λ v:s«312»}
    (app (app
      (λ v:z
        (handle
          (+ v:z (op:Get (op:Set 29)))
          ( ... )
          (return v:x (λ v:s v:x))))
      \textcolor{orange}{v:s«312»})
      \textcolor{orange}{v:s«312»}))
\textcolor{orange}{13)}
\end{Verbatim}
\begin{itemize}
  \item kolor \textcolor{orange}{pomarańczowy} -- redeks
\end{itemize}
\end{frame}

\begin{frame}[fragile]{Przykład}
\begin{Verbatim}[commandchars=\\\{\},numbers=left]
(app 
  (app
    (λ v:z«315»
      (handle
        (+ v:z«315» (op:Get (op:Set 29)))
        ( ... )
        (return v:x«318» (λ v:s«319» v:x«318»))))
    \textcolor{blue}{13})
  \textcolor{blue}{13})
\end{Verbatim}
\begin{itemize}
  \item kolor \textcolor{blue}{niebieski} -- wartości podstawione za zmienną
\end{itemize}
\end{frame}

\begin{frame}[fragile]{Przykład}
$\beta$-redukcja
\begin{Verbatim}[commandchars=\\\{\},numbers=left]
(app
  \textcolor{orange}{(app}
    \textcolor{orange}{(λ v:z«315»}
      (handle
        (+ \textcolor{orange}{v:z«315»} (op:Get (op:Set 29)))
        ( ... )
        (return v:x«318» (λ v:s«319» v:x«318»))))
    \textcolor{orange}{13})
  13)
\end{Verbatim}
\begin{itemize}
  \item kolor \textcolor{orange}{pomarańczowy} -- redeks
\end{itemize}
\end{frame}

\begin{frame}[fragile]{Przykład}
\begin{Verbatim}[commandchars=\\\{\},numbers=left]
(app
  (handle
    (+ \textcolor{blue}{13} (op:Get (op:Set 29)))
    ( ... )
    (return v:x«318» (λ v:s«319» v:x«318»))))
  13)
\end{Verbatim}
\begin{itemize}
  \item kolor \textcolor{blue}{niebieski} -- wartość podstawiona za zmienną
\end{itemize}
\end{frame}

\begin{frame}[fragile]{Przykład}
Wywołanie operacji \texttt{Set}
\begin{Verbatim}[commandchars=\\\{\},numbers=left]
(app
 (handle
  \textcolor{red}{(+ 13 (op:Get} \textcolor{orange}{(op:Set 29)}\textcolor{red}{))}
  ((op:Get ( ... ))
   (op:Set \textcolor{orange}{(v:s v:r (λ v:i«331» (app (app v:r 0) v:s)))}))
  (return v:x«332» (λ v:s«333» v:x«332»)))
 13)
\end{Verbatim}
\begin{itemize}
  \item kolor \textcolor{orange}{pomarańczowy} -- operacja i jej handler
  \item kolor \textcolor{red}{czerwony} -- przechwycony kontekst
\end{itemize}
\end{frame}

\begin{frame}[fragile]{Przykład}
\begin{Verbatim}[commandchars=\\\{\},numbers=left]
(app
  \textcolor{blue}{(λ v:i«343»}
    \textcolor{blue}{(app}
      \textcolor{blue}{(app}
        \textcolor{blue}{(λ v:z}
          (handle
          \textcolor{red}{(+ 13 (op:Get} \textcolor{blue}{v:z}\textcolor{red}{))}
          ( ... )
          (return v:x«332» (λ v:s«333» v:x«332»)))\textcolor{blue}{)}
        \textcolor{blue}{0)}
      \textcolor{blue}{29))}
  13)
\end{Verbatim}
\begin{itemize}
  \item kolor \textcolor{red}{czerwony} -- przechwycony kontekst
  \item kolor \textcolor{blue}{niebieski} -- wyrażenie stworzone przez redukcję
\end{itemize}
\end{frame}

\begin{frame}[fragile]{Przykład}
$\beta$-redukcja
\begin{Verbatim}[commandchars=\\\{\},numbers=left]
\textcolor{orange}{(app}
  \textcolor{orange}{(λ v:i«343»}
    (app
      (app
        (λ v:z
          (handle
          (+ 13 (op:Get v:z))
          ( ... )
          (return v:x«332» (λ v:s«333» v:x«332»))))
        0)
      29))
  \textcolor{orange}{13)}
\end{Verbatim}
\begin{itemize}
  \item kolor \textcolor{orange}{pomarańczowy} -- redeks
\end{itemize}
\end{frame}

\begin{frame}[fragile]{Przykład}
\begin{Verbatim}[commandchars=\\\{\},numbers=left]
(app
  (app
    (λ v:z
      (handle
      (+ 13 (op:Get v:z))
      ( ... )
      (return v:x«332» (λ v:s«333» v:x«332»))))
    0)
  29)
\end{Verbatim}
\end{frame}

\begin{frame}[fragile]{Przykład}
$\beta$-redukcja
\begin{Verbatim}[commandchars=\\\{\},numbers=left]
(app
  \textcolor{orange}{(app}
    \textcolor{orange}{(λ v:z}
      (handle
      (+ 13 (op:Get \textcolor{orange}{v:z}))
      ( ... )
      (return v:x«332» (λ v:s«333» v:x«332»))))
    \textcolor{orange}{0)}
  29)
\end{Verbatim}
\begin{itemize}
  \item kolor \textcolor{orange}{pomarańczowy} -- redeks
\end{itemize}
\end{frame}

\begin{frame}[fragile]{Przykład}
\begin{Verbatim}[commandchars=\\\{\},numbers=left]
(app
  (handle
    (+ 13 (op:Get \textcolor{blue}{0}))
    ( ... )
    (return v:x«332» (λ v:s«333» v:x«332»)))
  29)
\end{Verbatim}
\begin{itemize}
  \item kolor \textcolor{blue}{niebieski} -- wartość podstawiona za zmienną
\end{itemize}
\end{frame}

\begin{frame}[fragile]{Przykład}
Wywołanie operacji \texttt{Get}
\begin{Verbatim}[commandchars=\\\{\},numbers=left]
(app
  (handle
    \textcolor{red}{(+ 13} \textcolor{orange}{(op:Get 0)}\textcolor{red}{)}
    ((op:Get \textcolor{orange}{(v:i v:r (λ v:s (app (app v:r v:s) v:s)))})
     (op:Set ( ... ))
    (return v:x«332» (λ v:s«333» v:x«332»)))
  29)
\end{Verbatim}
\begin{itemize}
  \item kolor \textcolor{orange}{pomarańczowy} -- operacja i jej handler
  \item kolor \textcolor{red}{czerwony} -- przechwycony kontekst
\end{itemize}
\end{frame}

\begin{frame}[fragile]{Przykład}
\begin{Verbatim}[commandchars=\\\{\},numbers=left]
(app
  (handle
    \textcolor{red}{(+ 13} \textcolor{blue}{29}\textcolor{red}{)}
    ( ... )
    (return v:x«332» (λ v:s«333» v:x«332»)))
  29)
\end{Verbatim}
\begin{itemize}
  \item kolor \textcolor{blue}{niebieski} -- wynik po trzech krokach redukcji
\end{itemize}
\end{frame}

\begin{frame}[fragile]{Przykład}
Operacja bazowa
\begin{Verbatim}[commandchars=\\\{\},numbers=left]
(app
  (handle
    \textcolor{orange}{(+ 13 29)}
    ( ... )
    (return v:x«332» (λ v:s«333» v:x«332»)))
  29)
\end{Verbatim}
\begin{itemize}
  \item kolor \textcolor{orange}{pomarańczowy} -- operacja bazowa
\end{itemize}
\end{frame}

\begin{frame}[fragile]{Przykład}
\begin{Verbatim}[commandchars=\\\{\},numbers=left]
(app
  (handle
    \textcolor{blue}{42}
    ( ... )
    (return v:x«332» (λ v:s«333» v:x«332»)))
  29)
\end{Verbatim}
\begin{itemize}
  \item kolor \textcolor{blue}{niebieski} -- wynik operacji
\end{itemize}
\end{frame}

\begin{frame}[fragile]{Przykład}
Obsługa wartości
\begin{Verbatim}[commandchars=\\\{\},numbers=left]
(app
  (handle
    \textcolor{orange}{42}
    ( ... )
    \textcolor{orange}{(return v:x«332» (λ v:s«333» v:x«332»))})
  29)
\end{Verbatim}
\begin{itemize}
  \item kolor \textcolor{orange}{pomarańczowy} -- wartość i klauzula \texttt{return}
\end{itemize}
\end{frame}

\begin{frame}[fragile]{Przykład}
\begin{Verbatim}[commandchars=\\\{\},numbers=left]
(app 
  \textcolor{blue}{(λ v:s«405» 42)}
  29)
\end{Verbatim}
\begin{itemize}
  \item kolor \textcolor{blue}{niebieski} -- wyrażenie stworzone przez redukcję
\end{itemize}
\end{frame}

\begin{frame}[fragile]{Przykład}
$\beta$-redukcja
\begin{Verbatim}[commandchars=\\\{\},numbers=left]
\textcolor{orange}{(app}
  \textcolor{orange}{(λ v:s«405»} 42\textcolor{orange}{)}
  \textcolor{orange}{29)}
\end{Verbatim}
\begin{itemize}
  \item kolor \textcolor{orange}{pomarańczowy} -- redeks
\end{itemize}
\end{frame}

\begin{frame}[fragile]{Przykład}
\begin{Verbatim}[commandchars=\\\{\},numbers=left]
42
\end{Verbatim}
\end{frame}

\begin{frame}[fragile]{Przykład}
  Program:
  \VerbatimInput{03.rkt}
  \pause
  Ma typ:
  \pause
\begin{verbatim}
(t:g0 -> (op:Set (Num => Num)
          (op:Get (Num => Num) t:r14)) Num)
\end{verbatim}
\end{frame}

\begin{frame}[fragile]{Przykład}
Program:
\VerbatimInput{../algeff/test/06.rkt}
Oczekiwany wynik:
\texttt{42}
\end{frame}

\section{Podsumowanie}
\subsection{Sukcesy}
\begin{frame}{Sukcesy}
  \begin{itemize}
    \item Rachunek wraz z semantyką redukcyjną
    \begin{itemize}
      \pause
      \item Pełna redukcja wyrażeń
      \pause
      \item Wizualizacja krok po kroku
    \end{itemize}
    \pause
    \item Inferencja typów
    \pause
    \item Maszyna abstrakcyjna
    \begin{itemize}
      \pause
      \item Pełna redukcja wyrażeń
      \pause
      \item Wizualizacja krok po kroku
    \end{itemize}
    \pause
    \item Front-end ułatwiający pisanie wyrażeń
    \pause
    \item Integracja ze środowiskiem \Racket{}
  \end{itemize}
\end{frame}

\subsection{Wnioski i dalsza praca}
\begin{frame}{Wnioski i dalsza praca}
  Wnioski:
  \begin{itemize}
    \item Wizualizacja jest bardzo pomocna przy badaniu efektów algebraicznych
    \pause
    \item Biblioteka \Redex{} pozwala na rozwój rozbudowanego rachunku
  \end{itemize}
  \pause
  Dalsza praca:
  \begin{itemize}
    \pause
    \item Rozszerzenie rachunku o polimorfizm
    \pause
    \item Zbadanie własności rachunku za pomocą automatycznego generowania programów
  \end{itemize}
\end{frame}

\begin{frame}
  \centering
  Dziękuję za uwagę
\end{frame}

\begin{frame}
  \titlepage
\end{frame}

\end{document}
