\documentclass{beamer}

\mode<presentation>
{
  \usetheme{Warsaw}      % or try Darmstadt, Madrid, Warsaw, ...
  \usecolortheme{default} % or try albatross, beaver, crane, ...
  \usefonttheme{default}  % or try serif, structurebold, ...
  \setbeamertemplate{navigation symbols}{}
  \setbeamertemplate{caption}[numbered]
} 
\usepackage[polish]{babel}
\usepackage{amsmath}
\usepackage{amssymb}
\usepackage{fancyvrb}
\usepackage{fontspec}

\setmonofont{CMU Typewriter Text}

\newcommand{\Redex}{\texttt{PLT Redex}}
\newcommand{\Racket}{\texttt{Racket}}
\newcommand{\LC}{\(\lambda\)-calculus}

\title[A calculus with algebraic effects and~handlers]{Implementation of static and~dynamic semantics for a calculus with algebraic effects and~handlers using \Redex}
\author{Maciej Buszka}
\institute{Instytut Informatyki UWr}
\date{15.02.2019}

\begin{document}

\begin{frame}
	\titlepage
\end{frame}

% Uncomment these lines for an automatically generated outline.
% \begin{frame}{Outline}
% 	\tableofcontents
% \end{frame}

\begin{frame}{Plan prezentacji}
	\tableofcontents
\end{frame}

\section{Wstęp}
\subsection{Efekty algebraiczne}
\begin{frame}{Efekty algebraiczne}
	\begin{itemize}
		\pause
		\item Na wysokim poziomie pozwalają na:
		\begin{itemize}
			\pause
			\item Kompozycję wielu różnych efektów obliczeniowych
			\pause
			\item Definicję własnych efektów przez programistę
			\pause
			\item Rozdzielenie interfejsu i implementacji
		\end{itemize}
		\pause
		\item W użyciu:
		\begin{itemize}
			\pause
			\item Pisząc program/funkcję programista może użyć abstrakcyjnych operacji -- ich zbiór tworzy interfejs
			\pause
			\item Aby uruchomić taki program, musi on zostać umieszczony w wyrażeniu które te operacje obsłuży -- tworząc ich implementację
		\end{itemize}
	\end{itemize}
\end{frame}

\begin{frame}{Obsługa operacji}
	\pause
	W trakcie wykonania programu, gdy operacja zostaje wywołana:
	\begin{itemize}
		\pause
		\item Normalny tok obliczeń zostaje przerwany
		\pause
		\item Odnalezione zostaje odpowiednie wyrażenie obsługujące
		\pause
		\item Ma ono dostęp do:
		\begin{itemize}
			\pause
			\item Wartości przekazanej przy wywołaniu operacji
			\pause
			\item Funkcji reprezentującej resztę obliczenia które zostało przerwane
		\end{itemize}
		\pause
		\item Wyrażenie obsługujące może użyć wznowienia dowolnie wiele razy
	\end{itemize}
\end{frame}

\subsection{Przykład 1}
\begin{frame}[fragile]{Przykład 1}
  \pause
\begin{Verbatim}[commandchars=\\\{\}]
(handle
  (+ (op:Magic (- 2 1)) 1)
  ((op:Magic (v:x v:r (app v:r 42))))
  (return v:x -(v:x, 1))
\end{Verbatim}
\end{frame}

\begin{frame}[fragile]{Przykład 1}
\begin{Verbatim}[commandchars=\\\{\}]
(handle
  (+ (op:Magic \textcolor{orange}{(- 2 1)}) 1)
  ((op:Magic (v:x v:r (app v:r 42))))
  (return v:x -(v:x, 1))
\end{Verbatim}
\end{frame}

\begin{frame}[fragile]{Przykład 1}
\begin{Verbatim}[commandchars=\\\{\}]
(handle
  (+ (op:Magic \textcolor{blue}{1}) 1)
  ((op:Magic (v:x v:r (app v:r 42))))
  (return v:x -(v:x, 1))
\end{Verbatim}
\end{frame}

\begin{frame}[fragile]{Przykład 1}
\begin{Verbatim}[commandchars=\\\{\}]
(handle
  \textcolor{red}{(+} \textcolor{orange}{(op:Magic 1)} \textcolor{red}{1)}
  ((op:Magic \textcolor{orange}{(v:x v:r (app v:r 42))}))
  (return v:x -(v:x, 1))
\end{Verbatim}
\end{frame}

\begin{frame}[fragile]{Przykład 1}
\begin{Verbatim}[commandchars=\\\{\}]
\textcolor{blue}{(app}
  \textcolor{blue}{(λ v:z}
    \textcolor{blue}{(handle}
      \textcolor{red}{(+ }\textcolor{blue}{v:z} \textcolor{red}{1)}
      \textcolor{blue}{(...)}
      \textcolor{blue}{(return v:x -(v:x, 1))))}
  \textcolor{blue}{42)}
\end{Verbatim}
\end{frame}

\begin{frame}[fragile]{Przykład 1}
\begin{Verbatim}[commandchars=\\\{\}]
\textcolor{orange}{(app}
  \textcolor{orange}{(λ v:z}
    \textcolor{orange}{(handle}
      \textcolor{orange}{(+ v:z 1)}
      \textcolor{orange}{(...)}
      \textcolor{orange}{(return v:x -(v:x, 1))))}
  \textcolor{orange}{42)}
\end{Verbatim}
\end{frame}

\begin{frame}[fragile]{Przykład 1}
\begin{Verbatim}[commandchars=\\\{\}]
\textcolor{blue}{(handle}
  \textcolor{blue}{(+ 42 1)}
  \textcolor{blue}{( ... )}
  \textcolor{blue}{(return v:x«38» -(v:x«38», 1)))}
\end{Verbatim}
\end{frame}

\begin{frame}[fragile]{Przykład 1}
\begin{Verbatim}[commandchars=\\\{\}]
(handle
  \textcolor{orange}{(+ 42 1)}
  ( ... )
  (return v:x«38» -(v:x«38», 1)))
\end{Verbatim}
\end{frame}

\begin{frame}[fragile]{Przykład 1}
\begin{Verbatim}[commandchars=\\\{\}]
(handle
  \textcolor{blue}{43}
  ( ... )
  (return v:x«38» -(v:x«38», 1)))
\end{Verbatim}
\end{frame}

\begin{frame}[fragile]{Przykład 1}
  \begin{Verbatim}[commandchars=\\\{\}]
  (handle
    \textcolor{orange}{43}
    ( ... )
    \textcolor{orange}{(return v:x«38» -(v:x«38», 1))})
\end{Verbatim}
\end{frame}

\begin{frame}[fragile]{Przykład 1}
\begin{Verbatim}[commandchars=\\\{\}]
\textcolor{blue}{-(43, 1)}
\end{Verbatim}
\end{frame}

\begin{frame}[fragile]{Przykład 1}
\begin{Verbatim}[commandchars=\\\{\}]
\textcolor{orange}{-(43, 1)}
\end{Verbatim}
\end{frame}

\begin{frame}[fragile]{Przykład 1}
\begin{Verbatim}[commandchars=\\\{\}]
\textcolor{blue}{42}
\end{Verbatim}
\end{frame}

\subsection{Motywacja}
\begin{frame}{Motywacja}
	\begin{itemize}
		\pause
		\item Efekty algebraiczne są stosunkowo nowym zagadnieniem
		\pause
		\item Ich implementacja oraz semantyka (zarówno statyczna jak i dynamiczna) są nietrywialne
		\pause
		\item Istnieje już kilka języków programowania z efektami algebraicznymi
		\pause
		\item Brakuje jednak interaktywnej implementacji modelowego rachunku
	\end{itemize}
\end{frame}

\subsection{Cel}
\begin{frame}{Cel}
	Moim celem było zaprojektowanie oraz implementacja:
	\begin{itemize}
		\pause
		\item Rachunku oraz jego semantyki dynamicznej
		\pause
		\item Semantyki statycznej -- systemu typów
		\pause
		\item Maszyny abstrakcyjnej zgodnej z semantyka dynamiczną
	\end{itemize}
\end{frame}

\section{Rachunek oraz mikrojęzyk algeff}
\subsection{Rachunek}
\begin{frame}{Rachunek}
	\pause
	Rachunek który zaimplementowałem udostępnia:
	\begin{itemize}
		\pause
		\item Wyrażenia liczbowe z podstawowymi operacjami
		\pause
		\item Wyrażenia logiczne i warunkowe
		\pause
		\item Homogeniczne listy (dla dowolnego typu)
		\pause
		\item Funkcje anonimowe oraz rekurencyjne
	\end{itemize}
\end{frame}

\begin{frame}{Rachunek}
	Efekty algebraiczne są realizowane przez:	
	\begin{itemize}
		\pause
		\item Abstrakcyjne operacje
		\begin{itemize}
			\pause
			\item Nie muszą być zdefiniowane a priori
			\pause
			\item Przyjmują jeden argument, zwracają (być może) jedną wartość
		\end{itemize}
		\pause
		\item Wyrażenia obsługujące operacje
		\begin{itemize}
			\pause
			\item Dostęp do wznowienia reifikowanego jako funkcja
			\pause
			\item Semantyka tzw. głębokiej obsługi
			\pause
			\item Obsługa wielu operacji naraz
		\end{itemize}
	\end{itemize}
\end{frame}

\begin{frame}{System typów}
	\begin{itemize}
		\pause
		\item Implementacja tworzy funkcję odtwarzającą typ wyrażenia
		\pause
		\item Inferowane są tylko typy proste oraz operacje
		\pause
		\item Nie ma możliwości stworzenia funkcji polimorficznych
		\pause
		\item Ale system znajduje najogólniejszy (prosty) typ wyrażenia
	\end{itemize}
\end{frame}

\subsection{Przykład 2}
\begin{frame}[fragile]{Przykład 2}
	\pause
	Wyrażenie:
	\pause
	\VerbatimInput[firstline=4,lastline=6]{02.rkt}
	\pause
	Ma typ:
	\pause
\begin{verbatim}
(t:g0 -> (op:Set (Num => Num)
          (op:Get (Num => Num) t:r14)) Num)
\end{verbatim}
\end{frame}


\end{document}
